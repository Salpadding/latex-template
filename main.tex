\documentclass[a4paper]{article}
\usepackage{fontspec}
\usepackage{xeCJK}
\usepackage{indentfirst}
\setCJKmainfont{SimSong}
\setlength{\parindent}{2em}
\usepackage[super,square,sort&compress]{natbib}

%

\author{朱英杰}
\date{2024年12月}
\title{人脸识别综述}

\begin{document}
\maketitle

\section{引言}
人脸识别技术作为生物特征识别领域的一个重要分支,因其高精度和非侵入性特点,在安全监控、身份验证、智能交互等多个领域得到了广泛的应用。
随着深度学习技术的兴起,人脸识别技术取得了突破性进展,尤其是在人脸检测、识别和对齐等关键环节\cite{wanyonyi2022open} \cite{wang2022survey}。
本文旨在综述人脸识别技术的最新研究进展,包括人脸目标检测、人脸识别和人脸对齐三个子领域,分析其关键技术并探讨未来的发展趋势。

\section{人脸目标检测算法}

\subsection{传统人脸目标检测方法}

Viola-Jones 面部检测算法是一种实时的、基于机器学习的目标检测方法,由 Paul Viola 和 Michael Jones 于 2001 年提出\cite{wang2014analysis} \cite{priya2023viola} \cite{abdulhussien2022evaluation}
\cite{katke2020face}。这是第一个能够在实时应用中高效检测人脸的算法,因其快速性和可靠性成为计算机视觉领域的经典方法。

Viola-Jones 面部检测算法主要分为以下四个关键部分。

第一个部分是特征提取,这一部分使用了 Haar 的方法\cite{soo2014object}\cite{lienhart2002extended},Haar 特征值为白色矩形像素和减去黑色矩形像素和。Haar特征值反映了图像的灰度变化情况。例如:脸部的一些特征能由矩形特征简单的描述,如:眼睛要比脸颊颜色要深,鼻梁两侧比鼻梁颜色要深,嘴巴比周围颜色要深等。
单张图片可能生成数万个 Haar 特征,但只有少数与目标(如人脸)相关。针对大量的 Haar 特征计算,需要利用积分图进行快速计算。

第二个部分是特征选择,这一部分使用了 AdaBoost 算法\cite{ying2013advance} \cite{schapire2013explaining} \cite{hu2008adaboost}。AdaBoost 可以组合弱分类器生成强分类器。在 Viola-Jones 算法中,每个 Haar 特征对应一个弱分类器,AdaBoost 按照分类误差选择最优的弱分类器,并调整样本权重,组合多个弱分类器,形成一个强分类器,用于判断一个检测窗口是否可能包含目标。

第三个部分是级联分类器,在人脸目标检测中,大量检测窗口会被处理,大部分是负样本,也就是不包含目标的样本。对此,需要使用多个层级的强分类器,每一级都对检测窗口进行快速判断。
第一层是简单分类器,使用少量特征快速排除大多数负样本。后续的层再针对通过的样本进行更复杂的检测。

最后一个部分是滑动窗口和多尺度检测。滑动窗口是指把图片分成多个重叠的窗口,依次检测每个窗口是否包含目标。多尺度检测是指对图像进行缩放,生成多尺度版本,然后在每个尺度上应用滑动窗口检测,这么做的目的是检测不同大小的人脸。

\subsection{RetinaFace 算法}

RetinaFace算法是一种单阶段的人脸检测器\cite{deng2019retinaface} \cite{deng2020retinaface},它通过多任务学习和特征金字塔网络(FPN)\cite{lin2017feature} \cite{kim2018parallel} \cite{kirillov2019panoptic}来实现对不同尺寸人脸的像素级定位。

RetinaFace 将人脸目标检测中常用的三项任务整合到一个统一的框架中,它们分别是人脸检测,面部关键点定位以及 3D 面部网格回归。其中面部关键点包含眼睛,鼻子和嘴角。
这也被称为多任务学习。

RetinaFace 使用了深度学习技术,在神经网络的架构上采用了特征金字塔网络。这种网络结构在目标检测和图像分割任务中非常有效。FPN的核心思想是将高层特征(低分辨率、高语义信息)与低层特征(高分辨率、细节信息)结合,构建出一个多尺度的特征金字塔。这样做的目的是让网络能够在不同层次上捕获不同尺度的上下文信息,从而提升对不同大小目标的识别能力。

为了提高了对人脸检测精度,尤其是在处理不同尺度、姿态和遮挡条件下的人脸检测问题,RetinaFace 使用了 SSH(Single Stage Headless) 模块\cite{najibi2017ssh}结合 FPN,在检测锚点附近生成上下文信息,从而改善对小型和被遮挡人脸的检测。
SSH模块使用并行的三个卷积来进一步加强特征提取。这三个卷积分别是一个3x3卷积,一个二次3x3卷积和一个三次3x3卷积,没有使用到5x5或者7x7的卷积。通过这种方式,SSH模块能够增强模型的感受野,捕获更多的上下文信息,从而提高特征的表达能力。

RetinaFace在训练模型过程中针对不同的场景使用了以下几种损失函数。对于判断一个框是否包含人脸,使用了交叉熵损失(Softmax loss)。在优化人脸边界框定位以及关键点定位上面,使用了平滑 L1 损失函数(Smooth L1 Loss)。

\subsection{SCRFD 算法}

SCRFD(Sample and Computation Redistribution Face Detection)是一种高效的人脸检测算法\cite{guo2021sample},设计目标是平衡精度和计算效率,特别适合在边缘设备和移动平台上部署。
和RetinaFace相比,SCRFD 避免了多阶段检测器的复杂性,使其计算效率高,同时保持较高的检测精度。

SSCRFD算法的特点是采用了样本再分配(SR)和计算再分配(CR)策略。

样本再分配的主要思想是动态调整正负样本的分布,特别是在小目标检测任务中,更合理地分配样本以优化训练目标,例如通过将更多的训练样本关注于难以检测的目标(如小人脸),平衡不同尺度目标的检测能力。
其次,传统检测器在训练时容易出现正样本数量较少、负样本数量过多的问题。SR 策略通过重新分配样本权重,有效缓解了这一问题,降低了训练中正负样本比例失衡对模型的影响。

计算再分配的主要思想是根据目标尺度的不同,动态调整计算资源的分配,以提高网络的效率。CR 策略将计算资源更多分配到检测效果较差的区域或难以检测的特征点上,减少不必要的计算开销,使模型推理更高效。
CR 策略减少了对高计算资源的依赖,适合在资源受限的设备(如移动端)上部署 SCRFD 模型,同时保持较高的检测性能。

在网络架构的选择上,SCRFD使用了一种轻量级的神经网络架构,基于ResNet 和 MobileNet 的设计理念,并结合了FPN(Feature Pyramid Network),以实现多尺度人脸检测的高效性和准确性。

SCRFD模型因其高效性和高精度,在多个领域有广泛的应用案例。

SCRFD在安全监控领域可以用于复杂场景下的人脸快速、准确检测。无论是光线变化、遮挡还是表情变化等复杂情况,SCRFD都能展现出强大的检测能力,提升监控系统的智能化水平,为安全防范提供有力支持。

在人机交互领域,SCRFD可以应用于面部识别、表情识别等多个方面。通过高效、准确的人脸检测,SCRFD为用户提供更加自然、流畅的交互体验。例如,在智能家居系统中,SCRFD可以实现对用户面部的实时检测与识别,从而提供更加个性化的服务。

在智能相册管理领域,SCRFD可以自动识别并整理照片中的人脸信息。通过对照片中的人脸进行快速、准确的检测与识别,SCRFD可以帮助用户快速找到想要的照片或人物信息,提升相册管理的便捷性和高效性。

SCRFD在智能零售领域也有应用,可以在商场、超市等场所进行顾客流量统计和人脸支付,提升零售业的智能化水平。

\section{人脸识别算法}

\subsection{ArcFace算法}

ArcFace 的核心在于其使用的损失函数,在 softmax 损失函数的基础上,ArcFace提出了一个加性角度边界损失(Additive Angular Margin Loss),使得类别间的角度距离增大,从而显著增强了模型的判别能力\cite{deng2019arcface} \cite{deng2020sub} \cite{deng2019arcfacedisguised}。

为了提高模型对大规模噪声数据的鲁棒性,ArcFace进一步提出了子中心ArcFace,每个类别包含K个子中心,训练样本只需要靠近任意一个正样本子中心即可。

除了判别性特征嵌入外,ArcFace还探索了逆问题,即从特征向量映射回面部图像,无需训练额外的生成器或鉴别器,即可生成保留身份信息的面部图像。

与其他基于边界的softmax损失方法相比,如SphereFace和CosFace,ArcFace在多个测试集上均展现出更高的验证准确率\cite{firmansyah2023comparison}。

尽管ArcFace在多个基准测试中取得了优异的性能,但在处理极端条件下的人脸识别任务时,如遮挡、表情变化等,仍存在一定的局限性\cite{montero2022boosting}。未来的工作可以探索如何进一步提高模型在这些条件下的性能,以及如何将ArcFace应用于更广泛的应用场景中。

\subsection{PartialFC算法}

在人脸识别模型训练中,随着训练集中身份数量的增加,全连接层占用的内存和计算成本线性增加,这限制了模型的训练能力。
PartialFC(PFC)算法是一种针对大规模人脸识别训练问题提出的解决方案,旨在解决传统全连接层在处理大规模数据集时遇到的内存和计算成本问题\cite{an_2021_pfc_iccvw} \cite{an2022killing}。

传统的全连接层在面对大规模数据集时,会暴露出以下缺点。其一是大规模数据集的不同的类别对之间可能存在高余弦相似度,导致梯度混淆。每次迭代中,图片很少的身份可能导致负优化。再者是全连接层的存储和计算需求容易超出当前GPU的能力。

PartialFC算法在训练期间维护所有类别中心,但每次迭代只随机采样一小部分负类别中心来计算基于边界的softmax损失。这种方法在保持模型准确性的同时显著减少了计算需求,提高了训练效率,降低了类间冲突的概率,降低了GPU的内存使用和计算成本。具体实现是,PartialFC算法首先从每个GPU收集embeddings和标签,然后将组合的特征和标签分布到所有GPU。为了平衡每个GPU的内存使用和计算成本,为每个GPU设置了一个内存缓冲区。在每个GPU上,首先通过标签选择正类中心并放入缓冲区,然后随机选择一小部分负类中心填充缓冲区的其余部分。

PartialFC算法使得在一台机器上训练1000万个身份的数据集成为可能
。通过实现高效的分布式采样算法,该算法仅使用8个NVIDIA RTX2080Ti GPU就能完成数以千万计身份的分类任务。在多个数据集的实验结果表明,PartialFC算法只使用10\%的类来计算softmax,可以达到很高的准确性。PartialFC算法在WebFace数据集上进行了训练,该数据集包括4M、12M、42M个身份的数据
。实验结果表明,将PartialFC替换掉传统的FC层后,模型在这些数据集上的性能有所提升。

\subsection{VPL算法}

VPL(Variational Prototype Learning)算法的核心创新在于其对类别表示的方法。与传统的原型学习(例如softmax损失)不同,VPL将每个类别视为潜在空间中的一个分布,而不是一个单一的点。这种方法允许VPL在分类框架内模拟样本间的比较,鼓励SGD求解器更具探索性,同时提升性能 \cite{deng2021variational}。

VPL算法基于对训练后期收敛减缓的现象,直接将记忆中的特征注入到对应类别的原型中,以近似变分原型采样。这种方法不仅在计算上高效,而且在内存使用上节省,因为它不需要存储大量的样本特征,而是通过注入记忆中的特征来近似变分原型采样。这样可以显著提高模型训练的速度,同时保持模型的准确度。

VPL算法作为一个插件模块,可以被直接集成到现有的基于边界的或基于挖掘的softmax方法中,为深度人脸识别提供了一种正交的改进方法。


\section{人脸对齐算法}

\subsection{SDU-Net算法}

SDU-Net(Stacked dense u-nets)是一种基于U-Net的深度学习网络,专门用于医学图像分割任务,尤其是在视网膜血管分割领域表现出色\cite{guo2018stacked}。

SDU-Net算法通过引入注意力机制模块(Self-Attention, SEM)和双路上采样模块(Dual Path Upsampling, DPUS),旨在提升视网膜血管分割的效率和准确率。这种设计使得SDU-Net能够更好地捕捉图像的上下文信息,减少信息损失,并提高图像特征提取的准确性。

SDU-Net 算法包含了一个注意力机制模块和一个双路上采样模块。注意力机制模块能够充分捕捉图像的上下文信息,帮助模型更有效地忽略噪声和无关信息,减少无效计算,并通过聚焦于重要特征,在复杂背景下更精准地提取目标特征,提高分割结果的稳定性和准确性。
上采样模块用于提高图像分辨率,并补偿空间和信道信息,使得解码器部分能够更好地恢复图像细节,提高分割结果的精细度。

与原始U-Net相比,在DRIVE数据集上的实验中,SDU-Net的灵敏度提高了3.99\%,特异性提高了0.3\%。

SDU-Net作为一种新型的视网膜血管分割算法,已经在多个评价指标上显示出优越的性能。未来的研究方向可能包括算法在不同数据集上的泛化能力验证、计算复杂度和运行时间的优化,以及结合其他多模态医学影像数据进一步提升分割精度和鲁棒性。

\subsection{MTCNN算法}

随着人工智能技术的飞速发展,人脸识别技术在安全监控、身份验证、照片管理等多个领域得到广泛应用。MTCNN(Multi-task Cascaded Convolutional Networks)作为一种高效的人脸检测算法,因其出色的性能和实用性而受到广泛关注\cite{zhang2019multi} \cite{li2020face} \cite{zou2020multi}。
MTCNN由三个级联的网络组成,分别是P-Net(Proposal Network)、R-Net(Refine Network)和O-Net(Output Network)。
P-Net 用于生成初始的候选框,并对候选框进行分类和位置回归。
R-Net 对P-Net生成的候选框进行细化,进一步生成更精确的候选框。
O-Net 对R-Net生成的候选框进行最终的位置回归和分类,输出最终的人脸检测结果。

MTCNN 的工作流程分为三个阶段,在每个阶段后,MTCNN使用非最大值抑制技术来合并高度重合的候选窗口,避免重复检测
。在进入工作流程之前,需要把输入图像缩放到不同的尺度,构建图像金字塔。这是为了应对不同大小的人脸。

第一阶段是 P-Net 网络。它的主要任务是快速生成候选窗口。P-Net利用一个全卷积网络,生成一系列可能包含人脸的候选窗口,并输出每个候选窗口的边界框回归向量,用于后续对候选窗口的修正。

第二阶段是 R-Net 网络。它的作用是对P-Net生成的候选窗口进行高精度过滤和选择。R-Net接收P-Net输出的候选窗口作为输入,通过更复杂的卷积神经网络进行进一步的处理,输出更精细的分类和回归结果。

第三阶段是 O-Net 网络,也是最终生成人脸边界框和人脸关键点的网络。O-Net接收R-Net输出的候选窗口作为输入,通过功能更强大的卷积神经网络进行最终的处理,输出每个窗口中的人脸关键点位置,从而实现了端到端的人脸检测任务。

\section{结论和展望}

人脸目标检测算法作为人脸识别系统的首要步骤,其性能直接影响到后续任务的有效性。随着深度学习技术的发展,基于深度学习的人脸检测算法在准确性方面取得了显著提高。当前的人脸检测方法主要分为级联CNN架构、基于区域提议的CNN模型(如R-CNN和Faster R-CNN)、单阶段检测模型(如SSD)等
。面临的挑战包括在复杂环境下的检测精度、数据集的多样性和质量、以及算法的实时性要求
。未来的研究方向可能集中在提高对小人脸的检测鲁棒性、人脸遮挡处理、准确的轻量级模型开发、Few-Shot人脸检测以及减少人脸检测偏差等方面

深度人脸识别通过大规模数据集训练卷积神经网络获取更鲁棒的人脸表示,极大的提升了人脸识别性能。目前深度人脸识别的网络模型可分为深度卷积神经网络和轻量级网络,旨在使提取出的人脸特征对光照、姿态变化等因素具有鲁棒性,同时具备可区分性
。面临的挑战包括处理面部变化如大姿态、运动模糊、低照度等情况下的识别率较低,大规模训练数据存在噪声干扰,以及数据不平衡问题
。未来的发展方向可能包括设计对面部变化鲁棒的人脸识别方法,设计轻量级模型和更高效的模型训练方法,以及设计合理利用不平衡数据的训练方法。


人脸对齐技术旨在将不同角度、表情和姿态的人脸图像转换到统一的视角和姿态,从而便于后续的人脸识别。人脸对齐算法主要分为生成方法和判别方法,其中生成方法对人脸形状和外观都建立生成模型,而判别方法直接分别训练特征点检测器
。面临的挑战包括数据集的获取、算法的鲁棒性以及实时性要求
。未来的研究方向可能集中在提高对复杂条件下人脸对齐的鲁棒性和效率,以及开发更加高效和准确的对齐算法。

\renewcommand{\refname}{参考文献}
\bibliographystyle{unsrt}
\bibliography{main}


\end{document}
